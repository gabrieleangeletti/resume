% LaTeX file for resume
% This file uses the resume document class (res.cls)

\documentclass[margin]{res}
% the margin option causes section titles to appear to the left of body text
\textwidth=5.2in % increase textwidth to get smaller right margin
%\usepackage{helvetica} % uses helvetica postscript font (download helvetica.sty)
%\usepackage{newcent}   % uses new century schoolbook postscript font
\usepackage{fontawesome}
\usepackage{hyperref}

\begin{document}
    \name{Gabriele Angeletti\\[12pt]} % the \\[12pt] adds a blank line after name
    \address{London, UK  \\ (+44) 7803 056685}
    \begin{resume}
        \section{}
        \faEnvelope~\textbf{angeletti.gabriele@gmail.com} \\[5pt]
        \faGithub~\textbf{\url{https://github.com/blackecho}} \\[5pt]
        \faLinkedin~\textbf{\url{https://www.linkedin.com/in/gabriele-angeletti}} \\[5pt]
        \faSkype~\textbf{gabriele.angeletti1}

        \section{Experience}
            {\bf Research Engineer,} \href{http://www.bluevisionlabs.com}{Blue Vision Labs} \hfill 2018-01--Present\\
            Design and implementation of data pipelines to analyze the behavior of complex 3D mapping and localization systems.
            Build and improve distributed algorithms to perform 3D mapping.
            Side work on experimental projects involving the use of machine learning to improve the mapping and localization systems.

            {\bf Research Engineer intern,} \href{http://www.bluevisionlabs.com}{Blue Vision Labs} \hfill 2017-08--2017-12\\
            Involved in a project about replacing SIFT with deep learning to improve features description and matching.\\
	        Built tools to evaluate and monitor the performance of 3D mapping systems.\\
	        Contributed to the design of a data warehouse solution to efficiently store and retrieve hundreds of millions of records.

            {\bf Consultant,} \hfill 2014--2017\\
            Involved in different projects during university, mainly in full-stack web development,
            web crawling, and natural language processing.

            {\bf Student,} \href{http://www.innovactionlab.org/?lang=en}{InnovAction Lab} \hfill Spring 2013\\
            Entrepreneurship course about how to present business ideas directly to investors.

        \section{Education}
            Sapienza University of Rome, \hfill 2015--2017 \\
            M.Sc.\ in Engineering in Artificial Intelligence and Robotics, (English Degree) \\
            Final grade 110/110 \\
            Thesis: \textit{Adaptive Deep Learning through Visual Domain Localization \footnote{\url{https://github.com/blackecho/master-thesis}}}

            Sapienza University of Rome, \hfill 2011--2014 \\
            B.Sc.\ in Engineering in Computer Science and Automation, (Italian Degree) \\
            Final grade 106/110 \\
            Thesis: \textit{Statistical analysis of mobile apps reviews to improve users' QoE}

        \section{Engineering Skills}
        \begin{description}
            \item \textbf{Python}, proficient: SciPy ecosystem, PyTorch, TensorFlow, PySpark
            \item \textbf{JavaScript}, proficient: ES6, TypeScript, Angular
            \item \textbf{Golang}, some experience
            \item \textbf{RDBMS}, proficient: PostgreSQL, MySQL, AWS RedShift
            \item \textbf{Cloud technologies}: AWS (S3, EC2, ECS, EMR, Batch)
            \item \textbf{Data engineering}: Spark, Airflow, Luigi
        \end{description}

        \section{Projects}
        \begin{description}
            \item Deep Learning TensorFlow \footnote{\url{https://github.com/blackecho/Deep-Learning-TensorFlow}}:
                Ready to use implementations of various Deep Learning algorithms using TensorFlow.
        \end{description}

        \section{Publications}
        \begin{description}
            \item G. Angeletti, T. Tommasi, B. Caputo.
                "Adaptive Deep Learning through Visual Domain Localization".
                In: \textit{IEEE International Conference on Robotics and Automation (ICRA)}(2018) \footnote{\url{https://arxiv.org/abs/1802.08833}}
        \end{description}

        \section{Languages}
        \begin{description}
            \item Italian, native speaker
            \item English, fluent spoken and written skills
        \end{description}

        \section{Honors}
        \begin{description}
            \item Ranked 13 / 2189 among Python github users in Italy by \footnote{\url{http://git-awards.com/}} \hfill 2018
            \item Winner of Accenture Digital Hackathon Rome \hfill 2016
            \item NASA International SpaceApps Challenge \hfill 2015\\
                Local winner (Rome) \& Global winner for category Galactic Impact\\
                Project: CROPP---Cultures Risks Observation and Prevention
                Platform\footnote{\url{https://2015.spaceappschallenge.org/award/}}
        \end{description}

        \section{Activities \& Hobbies}
        \begin{description}
        	\item Applications of machine learning to visual and musical arts
            \item Open source projects
            \item Data visualization
        \end{description}
    \end{resume}
\end{document}
