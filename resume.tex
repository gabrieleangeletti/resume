% LaTeX file for resume 
% This file uses the resume document class (res.cls)

\documentclass[margin]{res} 
% the margin option causes section titles to appear to the left of body text 
\textwidth=5.2in % increase textwidth to get smaller right margin
%\usepackage{helvetica} % uses helvetica postscript font (download helvetica.sty)
%\usepackage{newcent}   % uses new century schoolbook postscript font 
\usepackage{fontawesome}
\usepackage{hyperref}

\begin{document}
    \name{Gabriele Angeletti\\[12pt]} % the \\[12pt] adds a blank line after name 
    \address{235 Brick Lane \\ E2 7ED, London, UK  \\ (+44) 7803 056685}
    \begin{resume}
        \section{}
        \faEnvelope~\textbf{angeletti.gabriele@gmail.com} \\[5pt]
        \faGithub~\textbf{\url{https://github.com/blackecho}} \\[5pt]
        \faLinkedin~\textbf{\url{https://www.linkedin.com/in/gabriele-angeletti}} \\[5pt]
        \faSkype~\textbf{gabriele.angeletti1}

        \section{Experience}
            {\bf Research Engineer,} \hfill 2018-01--Present\\
            \href{http://www.bluevisionlabs.com}{Blue Vision Labs}\\
            {\bf Research Engineer intern,} \hfill 2017-08--2017-12\\
            \href{http://www.bluevisionlabs.com}{Blue Vision Labs}\\
            {\bf Consultant,} \hfill 2014--2017
            \begin{itemize} \itemsep-2pt    	
                \item Freelance---Worked part-time at different projects during master degree.\\
                    Tech stack: PHP (Laravel), MySQL, JS (JQuery), Linux (SysAdmin)
                \item \href{http://www.ares2t.com/en/web/home.php}{Ares2t}---Implemented algorithms to crawl
                    mobile apps reviews from the web and to perform sentiment analysis (part of bachelor thesis) 
                \item \href{https://www.vvvvid.it}{VVVVID}---Developed a web app (Java/MySQL) to generate custom
                    data reports
            \end{itemize}

            {\bf Student,} \href{http://www.innovactionlab.org/?lang=en}{InnovAction Lab} \hfill Spring 2013
            \begin{itemize} \itemsep-2pt
                \item Entrepreneurship course about how to present business ideas directly to investors
            \end{itemize}

        \section{Education}
            Sapienza University of Rome, \hfill 2015--2017 \\
            M.Sc.\ in Engineering in Artificial Intelligence and Robotics, (English Degree) \\
            Final grade 110/110 \\
            Thesis: \textit{Adaptive Deep Learning through Visual Domain Localization}

            Sapienza University of Rome, \hfill 2011--2014 \\
            B.Sc.\ in Engineering in Computer Science and Automation, (Italian Degree) \\
            Final grade 106/110 \\
            Thesis: \textit{Statistical analysis of mobile apps reviews to improve users' QoE}
        
        \section{Engineering Skills}
        \begin{description}
            \item \textbf{Python}, proficient: SciPy ecosystem, TensorFlow, PyTorch
            \item \textbf{RDBMS}, proficient: MySQL, PostgreSQL, AWS RedShift
            \item \textbf{JavaScript}, proficient: ES6, React, D3
            \item \textbf{Golang}, beginner
            \item \textbf{Cloud technologies}: AWS EC2, AWS EMR, AWS S3, ElasticSearch
        \end{description}

        \section{Publications}
        \begin{description}
            \item G. Angeletti, T. Tommasi, B. Caputo.
                "Adaptive Deep Learning through Visual Domain Localization".
                In: \textit{IEEE International Conference on Robotics and Automation (ICRA)} (2018)
        \end{description}

        \section{Languages}
        \begin{description}
            \item Italian, native speaker
            \item English, fluent spoken and written skills
        \end{description}

        \section{Honors}
        \begin{description}
            \item Ranked 13 / 2189 among Python github users in Italy by \footnote{\url{http://git-awards.com/}} \hfill 2018
            \item Winner of Accenture Digital Hackathon Rome \hfill 2016
            \item NASA International SpaceApps Challenge \hfill 2015\\
                Local winner (Rome) \& Global winner for category Galactic Impact\\
                Project: CROPP---Cultures Risks Observation and Prevention
                Platform\footnote{\url{https://2015.spaceappschallenge.org/award/}}
        \end{description}

        \section{Activities \& Hobbies}
        \begin{description}
            \item Open source projects
            \item Data Visualization
            \item Applications of Neural Networks to Art
        \end{description}
    \end{resume} 
\end{document}
